\documentclass[11pt]{article}

\usepackage{amsmath,amsfonts,amssymb,amsthm}
\usepackage[margin=1in]{geometry}
\usepackage[colorlinks]{hyperref}
\usepackage{algpseudocode}
\usepackage{algorithmicx}
\usepackage{listings}
\usepackage{hyperref}
\usepackage{fancyhdr}
\usepackage{framed}
\usepackage{graphicx}
\usepackage{caption}
\usepackage{subcaption}
\usepackage{float}
\usepackage{enumerate}
\usepackage{tikz}
\usetikzlibrary{arrows}
\usetikzlibrary{shapes}

\newcommand{\C}{{\mathbb{C}}}
\newcommand{\F}{{\mathbb{F}}}
\newcommand{\R}{{\mathbb{R}}}
\newcommand{\Z}{{\mathbb{Z}}}
\newcommand{\e}[1]{\ensuremath{\times 10^{#1}}}

\newcommand{\ve}[1]{\boldsymbol{#1}}
\newcommand{\norm}[1]{\|{#1}\|}
\newcommand{\code}{\begingroup
  \catcode`_=12 \docode}
\newcommand{\docode}[2]{
	\begin{framed}
	\lstinputlisting[basicstyle=\ttfamily\scriptsize,language=#2,title=\underline{\texttt{#1}},tabsize=4,numbers=left]{#1}\end{framed}\endgroup}


\setlength\parindent{0pt}

\lstset{basicstyle=\ttfamily}
\lstset{language=C++,
                basicstyle=\ttfamily,
                keywordstyle=\color{blue}\ttfamily,
                stringstyle=\color{red}\ttfamily,
                commentstyle=\color{gray}\ttfamily}

\begin{document}

\thispagestyle{empty}
%%%%%%%%%%%%%%%%%%%%%%%%%%%%%%%%%%%%%%%%%%%%%%%%%%%%%%%%%

\title{Solving Small TSP Instances \\ ~ \\ \textit{You'll Clean That Up Before You Leave}}

\author{Shawn Brunstig, Ian Dimock, Yuguang Zhang \\ University of Waterloo}

\maketitle

%%%%%%%%%%%%%%%%%%%%%%%%%%%%%%%%%%%%%%%%%%%%%%%%%%%%%%%%%

\section{Introduction}
\section{Methods}
\subsection{Enumeration}
\subsection{Held-Karp 1-Trees}
\subsection{Bellman-Held-Karp Dynamic Programming}

The Bellman-Held-Karp algorithm was discovered independently by both Richard Bellman \cite{Bellman} and Michael Held and Richard Karp \cite{HeldKarp} in 1962.

\subsection{Subtour Branch and Cut}
\section{Testing Environment}
\section{Test Data}
\section{Discussion}
\section{Conclusions}

\nocite{*}
\bibliographystyle{plain}
\bibliography{references}

\section*{Appendix}
\subsection*{Result Tables}
\subsection*{Code}
\code{../onetree.h}{C++}
\code{../onetree.cpp}{C++}


%%%%%%%%%%%%%%%%%%%%%%%%%%%%%%%%%%%%%%%%%%%%%%%%%%%%%%%%%

\end{document}