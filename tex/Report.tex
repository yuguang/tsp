\documentclass[11pt]{article}

\usepackage{amsmath,amsfonts,amssymb,amsthm}
\usepackage[margin=1in]{geometry}
\usepackage[colorlinks]{hyperref}
\usepackage{algpseudocode}
\usepackage{algorithmicx}
\usepackage{listings}
\usepackage{hyperref}
\usepackage{fancyhdr}
\usepackage{framed}
\usepackage{graphicx}
\usepackage{caption}
\usepackage{subcaption}
\usepackage{float}
\usepackage{enumerate}
\usepackage{tikz}
\usetikzlibrary{arrows}
\usetikzlibrary{shapes}

\newcommand{\C}{{\mathbb{C}}}
\newcommand{\F}{{\mathbb{F}}}
\newcommand{\R}{{\mathbb{R}}}
\newcommand{\Z}{{\mathbb{Z}}}
\newcommand{\e}[1]{\ensuremath{\times 10^{#1}}}

\newcommand{\ve}[1]{\boldsymbol{#1}}
\newcommand{\norm}[1]{\|{#1}\|}
\newcommand{\code}{\begingroup
  \catcode`_=12 \docode}
\newcommand{\docode}[2]{
	\begin{framed}
	\lstinputlisting[basicstyle=\ttfamily\scriptsize,language=#2,title=\underline{\texttt{#1}},tabsize=4,numbers=left]{#1}\end{framed}\endgroup}


\setlength\parindent{0pt}
\setlength{\parskip}{3mm plus3mm minus2mm}

\lstset{basicstyle=\ttfamily,showstringspaces=false}
\lstset{language=C++,
                basicstyle=\ttfamily,
                keywordstyle=\color{blue}\ttfamily,
                stringstyle=\color{red}\ttfamily,
                commentstyle=\color{gray}\ttfamily}

\begin{document}

\thispagestyle{empty}
%%%%%%%%%%%%%%%%%%%%%%%%%%%%%%%%%%%%%%%%%%%%%%%%%%%%%%%%%

\title{Solving Small TSP Instances \\ ~ \\ \textit{You'll Clean That Up Before You Leave}}

\author{Shawn Brunsting, Ian Dimock, Yuguang Zhang \\ University of Waterloo}

\maketitle

%%%%%%%%%%%%%%%%%%%%%%%%%%%%%%%%%%%%%%%%%%%%%%%%%%%%%%%%%

\section{Introduction}

Many algorithms have been developed for the travelling salesman problem. Asymptotic behaviour can be analysed for these methods, but methods that are superior asymptotically may turn out to be slower for small problems due to large overhead costs. For example, enumerating all possible tours may take $O(n!)$ operations, but for a problem with only $n=5$ cities, could this be faster than the $O(n^2 2^n)$ Bellman-Held-Karp algorithm? This project explores such questions by implementing a few common algorithms for the travelling salesman problem, then comparing their run time for problems of varying sizes and types.

Section \ref{sec:methods} outlines the algorithms that were implemented for this project, including some implementation details and optimizations. Section \ref{sec:environment} describes the environment in which we tested the algorithms. Section \ref{sec:data} describes the types of problems we ran the algorithms over. Section \ref{sec:discussion} discusses the results. Finally, Section \ref{sec:conclusions} summarizes the project. The raw timing data as well as our code are contained in the appendix.

\section{Methods}
\label{sec:methods}

\subsection{Enumeration}

\subsection{Held-Karp 1-Trees}

In a pair of articles published in 1970 and 1971 \cite{OneTree70,OneTree71} Michael Held and Richard Karp proposed a method for constructing lower bounds for the traveling-salesman problem based on \emph{1-Trees}, and an accompanying ascent algorithm for solving the TSP.

If a graph $G$ has nodes labelled $1, 2, \dots, n$, then a 1-tree of the graph is formed by taking the node set and a subset of the edges such that the edges incident only with nodes $2, 3, \dots n$ form a tree, and there are 2 edges incident with node 1. We can see that with this definition, a tour of the graph is  a 1-tree.

While the TSP problem looks for a tour of minimum cost, we can similarly define the notion of a 1-tree of minimum cost (the 1-tree need not be rooted at node $1$). The core of Held and Karp's algorithm comes from the realization of what happens to the minimum cost tour and the minimum cost 1-tree when we alter the costs of the edges in the graph in a particular way.

Consider a vector $\ve{\pi} = (\pi_1, \pi_2, \dots, \pi_n)$. We consider what would happen if we update the costs of the edges $c_{i,j}$ as follows:
$$ c_{i,j} \leftarrow c_{i,j} + \pi_i + \pi_j $$

We note that since each node must have degree two in the optimal tour, this modification of the edge weights changes the optimal tour cost by $2 \sum\limits_i^n \pi_i$, but does not change the tour itself. The minimum cost 1-tree however is not guaranteed to remain the same on the modified graph.

Held and Karp showed that from these minimum cost 1-trees we can produce lower bounds for the TSP. By modifying $\ve{\pi}$ the ascent algorithm attempts to produce minimum cost 1-trees that look more and more like tours, increase the lower bounds until eventually the optimal tour is found.

The algorithm we implemented consists of three main components. The first is the computation of the minimum cost 1-trees. This is a fairly simple process which involves computing a minimum spanning tree on all but one node and then adding that nodes two cheapest edges to the MST to produce a 1-tree. By repeating this process on each node, one of the 1-trees generated is guaranteed to be of minimum cost. The second component is a simple iterative algorithm to modify $\ve{pi}$. This is done by increasing and decreasing respectively the nodes in the 1-tree which have more or less than degree 2. The last component is a branching method. We used the technique used in homework 2 of forcing a particular edge into or out of the 1-trees. We kept our branching nodes in a priority queue so that as soon as we pop a 1-tree from our queue that is a tour, we know it is the optimal one.

Over the development of the 1-tree algorithm, many optimizations were made. Many were programming details, but I've outlined a few more significant changes that together greatly increased performance. Firstly, we modified our Kruskal code from homework 1 to work on the subsets of our graph, instead of building a subgraph every time we wanted to build an MST. Another optimization was to sort all the edges of the modified graph (modified weights) only once. In this way we can choose the two edges incident to our \emph{ignored} node during our modified Kruskal's algorithm, and not need to re-sort when we consider the next node to ignore. Since we are branching on the smallest lower bound in our priority queue, we cannot declare a tour we've found optimal until it has been popped from the queue. We can however update our upper bound on the tour cost. This helps with the pruning that occurs during the $\ve{\pi}$ iterations. Lastly, a seemingly minor change was to update our maximum lower bound seen in the iterations of $\ve{\pi}$ when the bound was greater than \emph{or equal} to our current max. Intuitively this makes sense because the iterations strive to produce 1-trees more resembling tours, so even if the bound itself is not changing, the 1-tree producing that bound is hopefully more tour-like.

\subsection{Bellman-Held-Karp Dynamic Programming}

The Bellman-Held-Karp algorithm was discovered independently by both Richard Bellman \cite{Bellman} and Michael Held and Richard Karp \cite{HeldKarp} in 1962. It uses dynamic programming, which means it uses the solutions of smaller problems to create a solution to the larger problem.

\subsubsection{Algorithm}

The algorithm is best described with a recursive formula. First we pick some start city $x$. The choice of $x$ does not matter, as long as it stays fixed throughout the algorithm. If $t$ is a city, and $S$ is a set of cities that includes $t$ but not $x$, then we define $opt$ to be our dynamic programming table where $opt(S,t)$ is the minimum cost of a path that starts at $x$, ends at $t$, and passes through every city in $S$ exactly once (and does not pass through any cities not in $S$).

The base cases are simple to compute for this dynamic programming table. $opt(\{ t \} , t)$ is simply the distance from $x$ to $t$, since those are the only cities that can be in the path. We use $dist(x,t)$ to represent this distance. So we have

\[ opt(\{ t \}, t) = dist(x,t) \qquad \forall t \]

When $S$ has more than one city, we define $opt(S,t)$ recursively. Let $q \neq t$ be some city in $S$. We let $q$ be the second last city in the path, and we can find the optimal length by considering all possible values of $q$. We need the length of the shortest path to $q$, plus the distance from $q$ to $t$:

\[ opt(S,t) = \min_{q \in S, q \neq t} ( opt(S \setminus \{ t \} , q) + dist(q, t) ) \]

Finally, we need to calculate the optimal tour length. Following similar reasoning to the above formula, we know that the optimal tour must pass through every city in the problem, then return to $x$. So $x$ is the last city, and we let $t$ be the second last. If $N$ is the set of all cities in the problem except for $x$, then the optimal tour length $v^*$ is

\[ v^* = \min_{t \in N} ( opt(N , t) + dist(t, x) ) \]

The algorithm has a running time of $O(n^2 2^n)$, which gives it the best asymptotic bound that has been achieved so far \cite{bico}.

\subsubsection{Implementation}

We tested 3 implementations of the Bellman-Held-Karp algorithm. First was our own implementation, which was written before looking at any other existing implementations. Writing this implementation ensured we had a solid understanding of the algorithm, along with some implementation ideas that were not influenced by what others had done.

The code for this first implementation can be found in \texttt{bhk.h} and \texttt{bhk.cpp}, and can be run using the \texttt{-m 3} flag with our main program. It has a few unique implementation ideas compared to the other two, the first of which was to store the dynamic programming table in a C++ map, rather than an array. This made it easier to implement, since we did not need to calculate the size of the table beforehand or figure out how to translate a subset into a position in that array. However, this method is expected to be slower than one which uses an array, since an array has $O(1)$ lookup time while the map would require $O(\log(s))$ time where $s$ is the number of entries in the table.

Another unique idea was to generate all subsets of size $k$ at once using a recursive method. This idea increases the amount of memory required, so it may not be ideal for large problems.

Finally, a subset of cities was initially represented as a vector of integers. Later, the code was changed to use an integer where each bit represented whether or not a particular city was included in the subset. This change was inspired by the second implementation, and although no rigorous testing was done, it seemed to significantly improve the running time.

The second implementation was taken directly from \cite{bico}. It is the recursive code that is first presented when that chapter describes the Bellman-Held-Karp algorithm. This code was put into \texttt{bhk2.h} and \texttt{bhk2.cpp}, and can be run using the \texttt{-m 5} flag with our main program.

The final implementation of the algorithm is also from \cite{bico}. The chapter describes various optimizations to improve the running time, which are all used in this implementation. The original C code for this implementation is in \texttt{tour{\textunderscore}dp3.c}. This code was translated into C++ to fit with our program, so the version we used can be found in \texttt{bhk3.h} and \texttt{bhk3.cpp}. This implementation can be run with the \texttt{-m 6} flag.

\subsection{Subtour Branch and Cut}

The code we used for Branch and Cut was the code developed for homework 2, based on the board work and sample code seen in lectures. In particular, using subtour inequalities generated by finding disconnected components on the support graph generated from fractional LP solutions.

We made a few improvements from the code we submitted for homework 2. The first was branching on the \emph{1-side} before the \emph{0-side}, when including/excluding edges in the LP formulation. Secondly, we improved our nearest neighbour tour upper bound by computing the nearest neighbour tour leaving from each node. This change actually affects all the methods which used pruning. 

\section{Testing Environment}
\label{sec:environment}

\section{Test Data}
\label{sec:data}

In order to compare the algorithms, we need to test them on travelling salesman problems of varying sizes. This section describes the properties of the problems that we used for testing. We implemented two ways of generating problems, but due to time constraints we only analyzed one of them.

A few properties were common to all test problems. Firstly, all the problems were symmetric. This means that if $dist(a,b)$ is the distance from city $a$ to city $b$, then we have $dist(a,b) = dist(b,a)$. Secondly, all edge lengths were positive integers. This ensured that the algorithms would not be affected by floating point errors. Finally, all test problems were complete graphs, meaning there exists an edge between every pair of cities.

The method that we used to generate problems in this report first generated random points in a plane, which represented the locations of cities. These locations were truncated such that the $x$ and $y$ coordinates were integers. Furthermore, the problem generator ensured that each city location was unique. Finally, the length of an edge between two cities was calculated as the Euclidean distance between them, rounded to the nearest integer. This meant that all edge lengths were positive integers, and that the triangle inequality would hold. In other words, given three cities $a$, $b$, and $c$, we would have $dist(a,b) + dist(b,c) \geq dist(a,c)$.

The second method for generating problems (which is not analyzed in this report) did not guarantee the triangle inequality. Instead of generating random locations for the cities, random edge lengths were generated for every pair of cities. These edge lengths were positive integers. For some methods, such as enumeration and the Bellman-Held-Karp algorithm, we should expect similar performance results for problems generated by the two methods. Other algorithms, such as the 1-Tree method, can be sensitive to the distribution of edge lengths, so it was necessary to consider both types of problems.

Command line flags are provided to run these different problem-generating methods in our program. To run the first method, use the \texttt{-k} and \texttt{-b} flags to specify the number of cities and the grid size respectively. To run the second method (random edge lengths), use \texttt{-e} and \texttt{-l} to specify the number of cities and the maximum edge length respectively.

\section{Discussion}
\label{sec:discussion}

\section{Conclusions}
\label{sec:conclusions}


\nocite{*}
\bibliographystyle{plain}
\bibliography{references}

\section*{Appendix}
\subsection*{Result Tables}
\subsection*{Code}
\code{../onetree.h}{C++}
\code{../onetree.cpp}{C++}


%%%%%%%%%%%%%%%%%%%%%%%%%%%%%%%%%%%%%%%%%%%%%%%%%%%%%%%%%

\end{document}